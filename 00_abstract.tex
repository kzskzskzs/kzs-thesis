% アブストラクト
\begin{jabstract}

	本研究ではその表示端末の環境に応じたウェブ内の文章の可読性を向上させるための文章の整形ツール『ReaderLint』を提案する。一つのウェブサービス、SNS、ブログが多種多様なデバイスで閲覧可能である現代において、文章を打つ書き手側で行う校正は、
    読み手側のデバイスでの環境にはそぐわないレイアウトの文章を表示してしまう場合がある。
    その書き手と読み手の二者間の環境の違いによって生まれかねない文章レイアウトの齟齬を解消するべく、
    文章のリーダビリティの研究に基づき、読み手側のデバイスにて自動的に文章を整形し視認性、
    可読性を向上させるシステムを開発し、ReaderLintと名付けた。
	本論文でReaderLintの設計、実装、その応用例と考察について述べ、最後にまとめを行う。

\end{jabstract}


