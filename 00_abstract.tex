% アブストラクト
\begin{jabstract}
    本研究ではその表示端末の環境に応じたWeb内の文章の可読性を向上させるための文章の整形ツール『ReaderLint』を提案する. Webサービス,SNS,ブログ,チャットサービス等を多種多様なデバイスで閲覧可能である現代において,文章を打つ書き手側で行う校正は,読み手側のデバイスでの環境にはリーダビリティ(視認性)を損ねたレイアウトの文章を表示してしまうケースが発生する.このような文章レイアウトの問題を解消するべく,リーダビリティの研究に基づき,読み手側自身がその閲覧環境で視認性の高い文章レイアウトを自動改稿するシステムを開発し,ReaderLintと名付けた.本論文でReaderLintの設計,実装,その応用例と考察について述べ, 最後にまとめを行う.
\end{jabstract}


