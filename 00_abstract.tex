% アブストラクト
\begin{jabstract}
本研究ではその表示端末の環境に応じたWeb内の文章の可読性を向上させるための文章の整形ツール
『ReaderLint』を提案する. 1つのWebサービス,SNS,ブログが多種多様なデバイスで閲覧可能である現
代において,文章を打つ書き手側で行う校正は,読み手側のデバイスでの環境にはそぐわないレイアウトの
文章を表示してしまう場合がある.その要因により生まれかねない文章レイアウトの齟齬を解消するべく,
文章のリーダビリティの研究に基づき,読み手側自身がその環境でより視認性の高い文章を表示可能となる
システムを開発し,ReaderLintと名付けた.本論文でReaderLintの設計,実装,その応用例と考察について
述べ, 最後にまとめを行う.
\end{jabstract}


