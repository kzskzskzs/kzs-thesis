
\begin{bib}[100]

\begin{thebibliography}{}

  \bibitem{福田78}
  福田忠彦: 
    \newblock 図形知覚における中心視と周辺視の機能差, 
    \newblock テレビジョン学会誌,vol. 32, no. 6, pp. 492\textasciitilde498 (1978)

  \bibitem{神部86} 
    神部尚武 :
      \newblock 読みの眼球運動と読みの過程
      \newblock 国立国語研究所報告, vol. 85, pp. 29\textasciitilde66 (1986)

  \bibitem{神部89} 
  神部尚武 :
    \newblock 読みの眼球運動における一つの停留中の情報の受容範囲
    \newblock 国立国語研究所報告, vol. 96, pp. 59\textasciitilde80 (1989)

  \bibitem{中條99} 
  中條和光 :
  \newblock 「読み」の認知モデル:日本語文章の読みに関する実験的研究
  \newblock 協同出版 (1999)
   
  \bibitem{村田09} 
  村田匡輝, 大野誠寛, 松原茂樹 :
   \newblock 読みやすい字幕生成のための講演文への漸進的改行挿入
   \newblock 電子情報通信学会論文誌. D, 情報・システム, vol. 92, no. 9, pp. 1621\textasciitilde1631 (2009)

  \bibitem{小林14} 
  小林潤平, 関口隆, 新堀英二, 川嶋稔夫 :
    \newblock 分節間改行レイアウトを有する日本語リーダーの読み効率評価,
    \newblock 人工知能学会第 28 回全国大会 2014

  \bibitem{夏目}
  夏目漱石 :
  \newblock 吾輩は猫である
  \newblock 新潮文庫、1961年、p.5

  \bibitem{WCAG2.0}
  Web Content Accessibility Guidelines (WCAG) 2.0:
  \newblock \protect\url{https://waic.jp/docs/WCAG20/Overview.html#meaning}

\end{thebibliography}

\end{bib}
