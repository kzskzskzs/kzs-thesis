\newif\ifjapanese

\japanesetrue

\ifjapanese
  \documentclass[a4j,11pt]{jreport}
  \renewcommand{\bibname}{参考文献}
  \newcommand{\acknowledgmentname}{謝辞}
\else
  \documentclass[a4paper,11pt]{report}
  \newcommand{\acknowledgmentname}{Acknowledgment}
\fi
\usepackage[dvipdfmx]{graphicx}
\usepackage{thesis}
\usepackage{ascmac}
\usepackage{graphicx}
\usepackage{multirow}
\usepackage{url}
\usepackage{latexsym}
\usepackage{here}
\usepackage{listings,jlisting}

\lstset{%
  language={C},
  basicstyle={\small\ttfamily\footnotesize},%
  breaklines=true,%
  identifierstyle={\small},%
  commentstyle={\small\itshape},%
  keywordstyle={\small\bfseries},%
  ndkeywordstyle={\small},%
  stringstyle={\small\ttfamily},
  frame={tb},
  breaklines=true,
  columns=[l]{fullflexible},%
  numbers=left,%
  xrightmargin=0zw,%
  xleftmargin=3zw,%
  numberstyle={\scriptsize},%
  stepnumber=1,
  numbersep=1zw,%
  lineskip=-0.5ex%
}
\bibliographystyle{jplain}

\jclass  {卒業論文}
\jtitle  {ユーザーのリーダビリティを考慮した\\ブラウザ内部文章レンダリングの研究}
\juniv   {慶應義塾大学}
\jfaculty{環境情報学部}
\jauthor {小笹祐紀}
\jhyear  {01}
\jsyear  {2019}
\jkeyword{ブラウザ, リーダビリティ}
\jproject{増井俊之研究会}
\jdate   {2020年1月}

\begin{document}

\ifjapanese
  \jmaketitle
\else
  \emaketitle
\fi

% アブストラクト
\begin{jabstract}

	本研究ではその表示端末の環境に応じたウェブ内の文章の可読性を向上させるための文章の整形ツール『ReaderLint』を提案する。一つのウェブサービス、SNS、ブログが多種多様なデバイスで閲覧可能である現代において、文章を打つ書き手側で行う校正は、
    読み手側のデバイスでの環境にはそぐわないレイアウトの文章を表示してしまう場合がある。
    その書き手と読み手の二者間の環境の違いによって生まれかねない文章レイアウトの齟齬を解消するべく、
    文章のリーダビリティの研究に基づき、読み手側のデバイスにて自動的に文章を整形し視認性、
    可読性を向上させるシステムを開発し、ReaderLintと名付けた。
	本論文でReaderLintの設計、実装、その応用例と考察について述べ、最後にまとめを行う。

\end{jabstract}


  % アブストラクト。要独自コマンド、include先参照のこと
\tableofcontents  % 目次
\listoffigures    % 表目次
\listoftables    % 図目次

\pagenumbering{arabic}


\chapter{序論}
\label{chap:introduction}

本章ではWeb上におけるテキスト表示についての現状、Web上での書き手側の文章(日本語テキスト)の特色を述べ、
それを踏まえて本論⽂の目的、ならびに構成について述べる。

\newpage

\section{研究動機}

コンピューター技術の発展、およびスマートフォンやタブレット端末(モバイル端末)の普及に伴い、
人は読む文章を映し出されたものへと推移しつつある。また、ブログやSNSといったウェブサービスの台頭が加わり、
ウェブ上に文章や画像といったコンテンツを共有し、それを閲読、ないしリアクションをもらう機会に恵まれるようになった。
\\一方で、多種多様なデバイスが普及されたことにより、ブラウザやアプリケーションのデザインは表示形式は
その画面サイズに依存した表示(レスポンシブ)となる。よってその書き手側の表示環境と読み手側の表示環境は異なる場合がある。
以下の画像はAndroid,iPhone,PCブラウザ上でTwitterのタイムラインを表示した場合の画面である。
画面のサイズに依存して横幅が異なることで文章のレイアウト自体が変化するしていることが分かる。

\begin{figure}[H]
    \centering
    \label{fig:image1}
    \includegraphics[width=0.7\columnwidth]{image/01/img1.png}
    \caption[スマホの画面の比較] {スマホの画面の比較\footnotemark[1]}
\end{figure}

\footnotetext[1]{
    引用元 \protect\url{
        https://developer.android.com/training/multiscreen/screensizes?hl=ja.
    }
}

\begin{figure}[H]
    \begin{tabular}{cc}
        \begin{minipage}[t]{0.5\hsize}
            \centering
            \label{fig:image2}
            \includegraphics[keepaspectratio, width=0.5\columnwidth]{image/01/img3.jpg}
            \caption[Androidにおける] {スマホの画面の比較}
        \end{minipage}&

        \begin{minipage}[t]{0.5\hsize}
            \centering
            \label{fig:image3}
            \includegraphics[keepaspectratio, width=0.5\columnwidth]{image/01/img4.png}
            \caption[PCブラウザにおける] {PCブラウザにおける}
        \end{minipage}
    \end{tabular}
\end{figure}

また、同じデバイスを用いても多様なクライアントを想定するアプリケーションにおいて、
そのクライアントアプリのデザインの仕様によって異なる表示になる。
\\このとき、各デバイスのテキストエリアのサイズに従って打たれたテキストは、
別環境で表示されたときに崩れたレイアウトになることで、
可読性を損ねる場合があり、読み手側のストレスとなりえるケースが発生する。
\\このようなケースは書き手側が自身のWYSIWYGエディタや表示環境での
視認性を高めるため、意図的に改行を入れているため発生するため、
文章のレイアウトの型崩れはブラウザで修正することは困難であった。

つづいて文章レイアウトの問題として日本語の改行問題を取り上げる。
日本語の改行問題の詳細は『Budou:日本語のための自動折り返し制御ツール』\footnotemark[4]の
紹介ページから引用する。
\begin{quotation}
    ウェブページ上の日本語の文章は、行末に置かれると、単語の途中でも折り返されてしまうことがあります。
    皆さんも、以下のような文章を見たことがあるはずです。「新しい Android の世界へようこそ。」という見出しの
    「ようこそ」という単語が、「ようこ」と「そ」の間で折り返され、ひとまとまりの単語として認識しにくくなっています。
    このように、単語の途中で発生する折り返しは、文章の読みやすさを下げる一因です。
    \begin{figure}[H]
        \centering
        \label{fig:image4}
        \includegraphics[width=0.7\columnwidth]{image/01/img2.png}
        \caption[単語の途中で折り返しが発生している例] {単語の途中で折り返しが発生している例}
    \end{figure}
\end{quotation}

このようにWeb上の日本語の文章のレイアウトに関しては独自の問題がある。また、上記の画像では文末の字余りが例として出されていたが、
以下の画像における一行目の文末に「どこで」のいち「ど」が一文字だけが表示されるパターン、
4行目の文頭で「ニャーニャー」のうち「ャー」だけが表示されるパターンも、ひとまとまりの文意として認識しづらくなっている。
    \begin{figure}[H]
        \centering
        \label{fig:image5}
        \includegraphics[width=0.7\columnwidth]{image/01/img5.png}
        \caption[ひとまとまりの文意を損ねている例] {ひとまとまりの文意を損ねている例}
    \end{figure}

\section{本研究の目的}
本研究では書き手と読み手のデバイスの違いや、表示形式の違いに生まれる文章構成の誤差によって生まれる
文章レイアウトの崩れを解消し、読み手側に対してより可読性の高い文章を表示するようにフォーマットすることで
ユーザビリティを損ねなさせない表示を提供するシステムの開発を目的とする。

\section{本論文の構成}
 本論⽂は本章を含めた6章で構成される。
 \\第2章では、本研究に関連した研究、およびライブラリを紹介した上で問題点を整理する。
 \\第3章では、本論⽂で提案するシステムの基本構成と使い⽅について述べる。
 \\第4章では、ユーザーからのフィードバックをまとめ、本論⽂で提案する
システムの有効性と問題点について述べる。
 \\最後に、第5章で本論⽂のまとめと結論を述べる。
  % 本文1
\chapter{先行研究}
\label{chap:system}

この章ではモバイル端末における文章表示における読みやすさに関する研究論文を紹介する。次に既存のライブラリについて紹介し、
本研究の立ち位置を明確にする。

\newpage

\section{読む行為と認知に関する研究}

神戸(1994)は紙媒体での日本語の文章において、一つの注視点に停留している間に情報が収集される範囲は,
被験者によって個人差があるが,9文字から12文字の範囲であることを明らかにし、
加えて、注視点の平均的な移動距離は,3文字から5文字の間であることを示した。\cite{1}
また、中篠(1999)は日本語を読む行為において、人間の視点移動は文節単位で行われる、と述べた。\cite{2}

\subsubsection{視認性}
村田らはリアルタイムでの講演会での登壇者の音声から自動で字幕を生成する際の
視認性を向上させる研究の中で、話言葉はその意味のまとまりを考慮し改行を行うことにより、
視認性が向上することを報告している。\cite{3}

また、小林らは、村田らの話し言葉のみならず、書き言葉でも文章の折返し箇所にて文節を分断しない改行レイアウトを提案、
視線が停まる回数である停留数が減少することで、通常の改行レイアウト
より読み速度が向上することを明らかにした。 \cite{4}

\begin{figure}[H]
    \centering
    \label{fig:image6}
    \includegraphics[width=0.6\columnwidth]{image/02/img1.png}
    \caption[文節を分断しない改行レイアウト] {文節を分断しない改行レイアウト\footnotemark[1]}
\end{figure}
\footnotetext[1]{
    引用元 文節間改行レイアウトを有する日本語リーダーの読み効率評価
}

\subsubsection{可読性}

\section{ライブラリ}
文節をもとに改行レイアウトを行う既存のライブラリとして
Budou\footnotemark[2]とmikan.js\footnotemark[3]を紹介し、その特色を述べる。

\footnotetext[2]{
    \protect\url{https://developers-jp.googleblog.com/2016/10/budou.html}
}

\footnotetext[3]{
    \protect\url{https://github.com/trkbt10/mikan.js}
}

BudouはPythonで書かれたGoogle製の形態素解析APIであるCloud Natural Language API に
を用いたライブラリであり、テンプレートエンジンのフィルタやビルドツールのタスクとして
エディタに組み込んで利用する。
単語の境界判別と構文解析を行い文節を特定し、文節ごとにdisplay:inline-blockを指定したSPANタグで囲むことで、
文章の折り返し可能な位置で改行を加えることができる。
懸念点はAPIを叩くためGoogle Cloud APIの設定や通信が必要であるがゆえにローカル環境で完結しない点と、
そのリクエストが多くなると料金が発生する点にある。また、見出し語で用いることを前提としているため、
長文で行うことが困難である。

\begin{figure}[H]
    \centering
    \label{fig:image7}
    \includegraphics[width=0.6\columnwidth]{image/02/img2.png}
    \caption[Budouの動作イメージ] {Budouの動作イメージ\footnotemark[3]}
\end{figure}

mikan.jsは先述したBudouの問題点を解決するべくtrkbt10氏が開発したJavaScriptライブラリである。
簡易的な正規表現により文の区切れを特定し、Budouと同じ原理を用いて改行を加えている。
当ライブラリの利点はJavaScript製であるためにPython製のBudouよりも簡易的にフロントエンドに
組み込みやすく汎用性が高い点にあるとされる。一方の懸念点として簡易的な正規表現を利用したことにより、
大雑把な単位で文章を区切るため文章のレイアウトが損ねる点にある。またURLリンクや記号といった日本語以外の
文章に対しては精度が低いこともあげられる。また、Budouと同じ原理で改行を行うため、
同じく短文や見出し語といった限られた範囲での利用のみが想定されている。

\begin{figure}[H]
    \centering
    \label{fig:image8}
    \includegraphics[width=0.6\columnwidth]{image/02/img3.png}
    \caption[mikan.jsのデモ画面にて記号やURLを表示した例] {mikan.jsのデモ画面にて記号やURLを表示した例\footnotemark[4]}
\end{figure}

\newpage

ここで双方に共通する特色として、あくまでHTMLで文章を書く際に、文章を改行位置を自動的に選定するために用いる
ライブラリであることが挙げられる。つまりは人がSNSやブログのエディタ画面では用いられることは想定していない。

\subsection{Web上の文章の特色}
本節ではWeb上の文章の特色について、先述したように現代ではブログ、
SNSでは多様なユニコードによって紙媒体では不可能である記法について述べる。

\subsubsection{DOM}
ウェブブラウザ上で文章がが表示されるのはDOM上で行われる。
「一行の長さ」といった文章がどのようにレンダリング環境は
そのDOM、およびDOMにかけられているCSSの設定によって決まる。

\subsubsection{横書き}
外国基準でウェブブラウザはデザインは発展してきたのでPC上の
文章では横書きが主流である

\subsubsection{URLリンク}
文章内にURLリンクを貼り付けてよい。クリックした場合にはそのURLへと移動する。
近年ではInstagramのハッシュタグのようなものでもURLリンクを挿入されるケースがある。

\subsubsection{句点代わりの改行}
%おまえのせいで文章のレンダリングがずれる
日本語組版に基づく方法以外に文章の文意を切り分ける方法として、改行を多用するパターンが近年では増えている。
これはDOM内を左詰めに表示される文章において余白を用いる意識したものである。
また、DOMの横幅が大きくとられている故に一行あたりが長くなることで
視認性を損ねてしまうケースを見越し、書き手側が文章の途中で改行を入れる
ケースがある。前章の研究動機にて述べた書き手と読み手側のレイアウトのズレは、
特に「句点代わりの改行」によって発生するケースが散見される。

前者はSNS等にてよく見られる記法であり、箇条書き的な記法である。
後者は視認性に関するデザインを意識していないサービスにて利用される頻度が多い。



\section{まとめ}
まず、文章のリーダビリティの研究には視認性と可読性の二種類に大別され、
文章のレイアウトの改稿は前者の分野に属することを述べた。

そして本研究では文章のレイアウトを変更する、という視認性の操作を行い
可読性には依存しないものをとして取り扱う、とした。

次に小林ら文章の視認性をあげる方法として、文節間の区切りを改行箇所として
選択し表示する研究を紹介し、その有意性を述べた。
また、MikanJSやBudouのように文節ごとにSPANタグを入れ込むことで、
PCの画面のサイズに対してレスポンシブに文節間の間が可能になるライブラリを紹介した。
これらにより文節で区切ることが可能になりつつある。

一方で、このような校正は書き手側が文章を書いているタイミングで設定する必要がある。
つまりは書き手側のためのライブラリであり、すでに書き終え、送信した文章に対して、
つまりは読み手側の環境に対してレイアウトを変更するツールするライブラリ、
サービスは散見されなかった。


  % 本文2
\chapter{実装}
\label{chap:system}

本章では「Web上の文章を成形すること」の定義を明確にした上で、
その実装形態である「ReaderLint」の要件と設計について述べる。

\section{システムの定義}

2.4にて示した文章の成形問題を解決するための要件は以下の3つであると仮定する。
  
\begin{enumerate}
	\item 行頭、行末に並ぶべきではない禁則が発生した場合に補正を行う(禁則処理)
	\item 書き手側がつけた改行が文章の段落位置が読み手側から見たときに可読性を損ねていた場合には操作を行う。
	\item 上記3つを書き手側の意図(改行による段落生成など)、文意を損ねずに行う
\end{enumerate}


\section{基本構成}

\subsection{DOMに入る文字数の計測}
\subsubsection{canvas} 
\subsection{文章解析}
\subsubsection{AST}
\subsection{文節の解析}
\subsubsection{tinySegmenter}
  
  % 本文3
\chapter{}
本章ではReaderLintを試験的に運用して得られたフィードバックや問題点をもとに考察を行う

\newpage

\section{評価}

本章では実装したReaderLintを実際に運用し,利用したユーザーからのフィードバックを評価として述べる.

\subsection{筆者の運用,利用における意見}
筆者を含め,ReaderLintを体験してもらった.主にReaderLintを用いた場面は
Twitter(TweetDeck)\footnotemark[1]\label{twitter},
Note \footnotemark[2]\label{note},
はてなブログ \footnotemark[3]\label{hatenablog} 
といったテキストを読む行為が主体であるWebサービスである.
以下は今回体験したユーザーからの文章レイアウトを変更することに関してのフィードバックの一例である.

\footnotetext[1]{
    引用元 \protect\url{
        https://tweetdeck.twitter.com/
    }
}
\footnotetext[2]{
    引用元 \protect\url{
        https://note.com/
    }
}
\footnotetext[3]{
    引用元 \protect\url{
        https://hatenablog.com/
    }
}
\subsubsection{文章を見失わずに読める}
文章のつながりが分かりやすいため,文末と次行の文頭の前後関係の判断が行いやすく,
文章の文頭を見失うことが少なかったという意見を得た.
これは第2章にて紹介した小林らによる論文でも有意性を示していた視認性から伺える評価であると考えられる.

つづいて,DOMサイズが大きい,つまりは横幅の大きいゆえに一行の文字数が長くなるテキストに対して,
横幅が900pxを超えたDOMに対し横幅を900px以下に収まるよう処理を加えたところ,視認性が向上したように思えた,
という意見を得た.これは一行における文字が多いため,元の文章自体の視認性が低かった文章に対して
視野内に収まりやすいレイアウトとなったことが要因である,と推測する.
これも第2章で述べられた行末と次行の行頭を追うためのサッカードの距離が短くなったため,文章を追うことができるようになった,
という眼球運動のフォローとなった結果ではないか,と考える.

\subsubsection{両端揃えでない文章への違和感}
対象となる横書きの文章文章が長いほど,左端のみならず右端も揃った,つまりは従来の
横組版に沿った両端揃えではない文章レイアウトに違和感を覚える,という意見を得た.
改行を意図的に行なうレイアウトと右詰されていない文章とを両立するのは難しく,
設定する場合にはCSSを用いたスタイル変更を用いることになる.(text-align-all:justify)
この方法による両端揃えは両端揃えの調整が必要になった際,字間を大きくすることで行末まで詰める仕様になっている.
%画像あったほうがいいかも
\begin{figure}[H]
    \centering
    \label{fig:ryohashi}
    \includegraphics[width=0.5\columnwidth]{image/04/img1.png}
    \caption[両端揃えを行なった文章レイアウト] {両端揃えを行なった文章レイアウト}
\end{figure}

そのため一行に含まれる文字数が少ない場合には字間を大きくとってしまい,文字と文字の間が開いて視認性を損ねる
可能性が生じてしまう.また,アクセシビリティの観点からこのような文章の字間を調整することで両端揃えを行う処理は,字間が不規則になることで失読症患者など
認知問題を抱えた読み手にとって視認性を損ねるレイアウトである,という報告がW3Cのガイドラインとしても掲げられている.\cite{WCAG2.0} 
ReaderLint側でも両端揃えの文章レイアウトに慣れているユーザーにとって,ReaderLintによって生まれる右端の凹凸がどれほど大きくなれば違和感を持つのか,その行末までの距離,
従来の改行レイアウトと凹凸を少なくした改行レイアウトとを比較した上,視認性がどこまで失われたと感じられるのか,を踏まえた上で今後の改良点
としたい.

\subsubsection{文節の切り方によって二つの意味が生じる文章}
これは文節で文章を区切る,という改行方法はレイアウト側によって文意を規定してしまう場合が起こりえる,という意見があったことで考察を行う.
例えば「この先生き残るにはどうすればいいか」というような例文があった場合に
「この/先/生き/」という区切り方と「この/先生/」という区切り方とでは文意が大きく異なってしまう.
このような日本語における形態素解析や分かち書き処理精度の向上は自然言語処理研究も話題に上がり,精度の向上を試みる研究が多くなされている.
ReaderLintにて使用しているTinySegmenterでこの文章の分かち書きを行なったテストでは,
以下画像のように後者の区切り方となった.
\begin{figure}[H]
    \centering
    \label{fig:seid}
    \includegraphics[width=0.7\columnwidth]{image/04/img2.png}
    \caption[TinySegmenterを用いた日本語分かち書きの精度] {TinySegmenterを用いた日本語分かち書きの精度}\footnotemark[1]
\end{figure}

\footnotetext[1]{使用サイト参照:
	\protect\url{http://chasen.org/~taku/software/TinySegmenter/}
}
このように書き手側の文意に問わず,ReaderLintでは日本語分かち書きに沿った改行を行うため,読み手が文意の誤読を起こしかねない.
この解決方法としてはより精度の高い分かち書きライブラリを用いることで精度を上げる,
辞書データ型の日本語分かち書きライブラリを用いることが挙げられる.今回はJavaScript製で辞書データを用いた
形態素解析ライブラリであるkuromoji.jsを用いて,先ほど入力したテキストを分かち書きすると以下のような結果となった.
\begin{figure}[H]
    \centering
    \label{fig:seid2}
    \includegraphics[width=0.7\columnwidth]{image/04/img3.png}
    \caption[kuromojiを用いた分かち書きの精度] {kuromojiを用いた分かち書きの精度}\footnotemark[2]
\end{figure}

\footnotetext[2]{使用サイト参照:
	\protect\url{https://azu.github.io/text-map-kuromoji/}
}

画像のように辞書データ依存の日本語分かち書きは文章として,文章に違和感のない分かち書きに成功している.
この一方で,第3章でも述べたように辞書データ型は辞書に載っていない単語,
たとえば,近年リリースされたコンテンツのタイトル名称等といった未知語を処理する際には不安定な結果を出力する場合がある.
このような未知語に対しての処理が対策が必要である.
一例をあげると,作品のタイトルを囲む際に用いられる二重括弧(例:『名探偵コナン』)で囲まれた文字列は1つの固有名詞として扱う,
などの日本語分かち書きする処理に対し,柔軟なルールを実装するといった配慮が必要だろう.
また,ユーザー個人が多用するが辞書として登録されていない単語等には新たに辞書データに単語を登録し学習することで
未知語だった単語の検出率を上げる,といった方法も有効である.

\section{まとめ}
ReaderLintを運用したユーザーのフィードバック,およびそれを踏まえた考察を行なった上で改善策を述べた.
文節間改行レイアウトに関してはその視認性はすでに研究として着目され,その有用性を示されており,実際に視認性がよくなった,
というフィードバックが多かった.その一方でその対象となる文章が長いほど,従来の横組版とは異なる両端揃えでないことに
違和感を感じるフィードバックもあがっていた.また,自然言語処理系の特有の問題である分かち書きの精度の問題から,文章の意図を異なる解釈で
改行を行い,読み手に誤った認識を与えてしまう可能性も示唆された.  % 本文4
\chapter{結論}

本章では本研究の総括を行なう.

\newpage

\section{研究の成果}
本研究ではディスプレイ内に表示される文章レイアウトの視認性はデバイスの画面サイズの差異により大きく損ねてしまう
可能性があること,また,Web上の日本語の文章は横組版のルールに沿って両端揃えになるように左詰めされた
文章レイアウトによって文意を分断した折返しによりユーザビリティが損なわれるケースを解消するシステムとして,
『ReaderLint』を提案した.

次に,関連研究では,文章を読む人間の認知処理や処理を紹介した上で,
日本語を読む際の動作として,視線は文節間を移動する傾向があることを示した上で,文章の視認性という観点に立った上で,
文節と文節の間を改行位置とする文節間改行レイアウトに着目し,短文のみならず文章全体のレイアウトを変える,
というライブラリに有用性があるのではないか,と予想した.

設計と実装では.これらの問題を踏まえた上,ブラウザ上で動作するReaderLintの基本機能とその実装方法についてを述べた.

考察では,ReaderLintを運用し,得られたフィードバックから本研究の有用性と現状の問題点を提示し,以降の改善点を示した.

\section{総括}
本研究ではWeb上で表示される日本語の文章レイアウトが抱える問題を解消する『ReaderLint』を開発した.

今後は考察で述べた問題点の改善や,稼働するデバイスの拡張,視認性を向上させるための更なる機能実装等を行なっていく.

\include{06}

\begin{acknowledgment}
    学部四年次という研究室に遅れて参加した私に対し,自由に研究を進める環境,および指導を賜りました増井俊之教授に心から感謝を申し上げます.
    同研究室の方々からは研究に対する指針やアドバイス,まだ不慣れであったプログラミングに対して手法やツールや
    ReaderLintのフィードバックを本研究をつづけることができました.ここに感謝を申し上げます.
    また, ReaderLintを使用しフィードバックをくださった先輩方や友人にも深く感謝いたします.
    最後に休学を含めた私の五年間の学生生活を支えてくださった方々に心からの感謝の意を表し,謝辞とさせていただきます.
\end{acknowledgment}


  % 謝辞。要独自コマンド、include先参照のこと

\begin{bib}[100]

\begin{thebibliography}{}

  \bibitem{福田78}
  福田忠彦: 
    \newblock 図形知覚における中心視と周辺視の機能差, 
    \newblock テレビジョン学会誌,vol. 32, no. 6, pp. 492\textasciitilde498 (1978)

  \bibitem{神部86} 
    神部尚武 :
      \newblock 読みの眼球運動と読みの過程
      \newblock 国立国語研究所報告, vol. 85, pp. 29\textasciitilde66 (1986)

  \bibitem{神部89} 
  神部尚武 :
    \newblock 読みの眼球運動における一つの停留中の情報の受容範囲
    \newblock 国立国語研究所報告, vol. 96, pp. 59\textasciitilde80 (1989)

  \bibitem{中條99} 
  中條和光 :
  \newblock 「読み」の認知モデル:日本語文章の読みに関する実験的研究
  \newblock 協同出版 (1999)
   
  \bibitem{村田09} 
  村田匡輝, 大野誠寛, 松原茂樹 :
   \newblock 読みやすい字幕生成のための講演文への漸進的改行挿入
   \newblock 電子情報通信学会論文誌. D, 情報・システム, vol. 92, no. 9, pp. 1621\textasciitilde1631 (2009)

  \bibitem{小林14} 
  小林潤平, 関口隆, 新堀英二, 川嶋稔夫 :
    \newblock 分節間改行レイアウトを有する日本語リーダーの読み効率評価,
    \newblock 人工知能学会第 28 回全国大会 2014

  \bibitem{夏目}
  夏目漱石 :
  \newblock 吾輩は猫である
  \newblock 新潮文庫、1961年、p.5

  \bibitem{WCAG2.0}
  Web Content Accessibility Guidelines (WCAG) 2.0:
  \newblock \protect\url{https://waic.jp/docs/WCAG20/Overview.html#meaning}

\end{thebibliography}

\end{bib}
  % 参考文献。要独自コマンド、include先参照のこと

\end{document}