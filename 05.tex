\chapter{結論}

本章では本研究の総括を行なう.

\newpage

\section{研究の成果}
本研究ではディスプレイ内に表示される文章レイアウトの視認性はデバイスの画面サイズの差異により大きく損ねてしまう
可能性があること,また,Web上の日本語の文章は横組版のルールに沿って両端揃えになるように左詰めされた
文章レイアウトによって文意を分断した折返しによりユーザビリティが損なわれるケースを解消するシステムとして,
『ReaderLint』を提案した.

次に,関連研究では,文章を読む人間の認知処理や処理を紹介した上で,
日本語を読む際の動作として,視線は文節間を移動する傾向があることを示した上で,文章の視認性という観点に立った上で,
文節と文節の間を改行位置とする文節間改行レイアウトに着目し,短文のみならず文章全体のレイアウトを変える,
というライブラリに有用性があるのではないか,と予想した.

設計と実装では.これらの問題を踏まえた上,ブラウザ上で動作するReaderLintの基本機能とその実装方法についてを述べた.

考察では,ReaderLintを運用し,得られたフィードバックから本研究の有用性と現状の問題点を提示し,以降の改善点を示した.

\section{総括}
本研究ではWeb上で表示される日本語の文章レイアウトが抱える問題を解消する『ReaderLint』を開発した.

今後は考察で述べた問題点の改善や,稼働するデバイスの拡張,視認性を向上させるための更なる機能実装等を行なっていく.
