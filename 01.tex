\chapter{序論}
\label{chap:introduction}

本章ではWeb上におけるテキスト表示についての現状を述べ、
それを踏まえて本論⽂の目的、ならびに構成について述べる

\section{研究動機}

コンピューター技術の発展、およびスマートフォンやタブレット端末(モバイル端末)の普及に伴い,
人が読む文章は本に印刷された活字からディスプレイに映し出されたものへと推移しつつある。
また、ブログやSNSといったウェブサービスの台頭が加わり、ウェブ上に文章や画像といったコンテンツを共有し、
それを閲読、ないしリアクションをもらう機会に恵まれることとなった。
\\一方で、普及したモバイル端末はあらゆる形状とサイズで提供されているため、
そのディスプレイ表示の設定は画面サイズ応じて伸縮、またはデザイン自体を変えた表示(レスポンシブデザイン)になる。
そのため、文章を表示する領域は画面サイズに依存して変化することになる。

\section{本研究の目的}

本研究では上記の問題書き手と読み手のデバイスの違いにより生まれる文章構成の誤差)を解消し、
読み手側のテキストをより高い可読性の文章になるように成形することが目的である。

\section{本論文の構成}


 本論⽂は本章を含めた6章で構成される。
 第2章では、本研究の背景をより詳細に分析し、Web上のテキストの特徴、問題点を整理する。
 第3章では、本論⽂で提案するシステムの基本構成と使い⽅について述べる。
 第4章では、本論⽂で提案するシステムの詳細な実装について述べる。
 第5章では、ユーザーからのフィードバックをまとめ、本論⽂で提案するシステムの有効性と問
題点について述べる。
 最後に、第6章で本論⽂のまとめと結論を述べる。