\chapter{考察}

\section{評価}

本章では実装したReaderLintを実際に運用し、利用したユーザーからの評価をもとに考察を述べる。


\subsection{筆者の運用、利用における意見}
筆者を含め、複数人にてReaderLintを一週間ほど利用した。主にReaderLintを用いた場面は
Twitter(TweetDeck)\footnotemark[1]\label{twitter}、
Note \footnotemark[2]\label{note}、
はてなブログ \footnotemark[3]\label{hatenablog} 
といったテキストが主体となるウェブサービスである。以下は今回体験したユーザーからの感想、意見である。


\footnotetext[1]{
    引用元 \protect\url{
        https://tweetdeck.twitter.com/
    }
}

\footnotetext[2]{
    引用元 \protect\url{
        https://note.com/
    }
}

\footnotetext[3]{
    引用元 \protect\url{
        https://hatenablog.com/
    }
}

文章のつながりが分かりやすいため、文末と次行の文頭の前後関係の判断が行いやすく、
文章の文頭を見失うことが少なかった、というものである。
これは第二章にて紹介した小林らによる論文でも述べられている評価である。

次に利用した感想としてはDOMサイズが大きい、
つまりは横幅の大きいゆえに一行の文字数が長くなるテキストに対して、
一層に視認性が向上したように思えた、というものがある。

これは一行における文字数が多すぎる故に、元の文章自体の視認性がもともと低かった場合に対して、
改行処理、および横幅が900pxを超えたDOMに対し横幅を900px以下に収まるよう
処理を加えたことが要因である、と推測する。

%画像あったほうがいいかも
また、横書きの文章に対して左端がそろっているものの方が読みやすい、
または違和感を感じられる、という双方の意見が上がった。前者は空白があること、
凹凸があることで文章を見失わずに読むことができる、という感想をもらった。
一方の後者には右端まで詰める文章レイアウトは本来は活字媒体の横書組版のルールとして
組み込まれているものであるため違和感を感じられる、という意見であった。

%画像あったほうがいいかも
同時に文節で文章を区切る、という改行方法は文意をレイアウト側によって、
規定してしまうことが起こりえるのではないか、という意見もあった。
例えば以下のような例文にReaderLintを使用する。このとき、文章ではTinySegmenterの
決壊に依存した分け方となる。これは文章の形態素解析の精度が問題となるが改行位置によって
書き手側の文意を捻じ曲げてしまう可能性があるのは憂慮するべきである。

%画像あったほうがいいかも
